\section{Wiederholung}
\label{sec:recap}

\subsection{Lösen von ODEs}
\begin{frame}
    \frametitle{\insertsubsection}
    Die folgende ODE beschreibt Protein-Synthese mit Degradation.
    \[\dot{x} = f(x, t) = \alpha - \beta x\]
    Hier sind $\alpha$ die Synthetisierungsrate und $\beta$ die Degradationsrate.
    \pause Fragen:
    \begin{enumerate}[<+->]
        \item Welche Einheiten haben $\alpha,\beta$
        \item Was erwarten wir für ein Verhalten? Warum?
        \item Wie gehen die Anfangswerte ein?
        \item Gibt es einen Gleichgewichtszustand? Ist er positiv?
    \end{enumerate}
\end{frame}


\begin{frame}[fragile]
    \begin{minted}[linenos, fontsize=\scriptsize, escapeinside=||]{python}
|\pause|from scipy.integrate import odeint
import matplotlib.pyplot as plt
import numpy as np

def f(x, t, a, b):
    return a - b*x

if __name__ == "__main__":
    tstart = 0.0
    tend = 10.0
    y0 = 0.0
    a = 0.1
    b = 10.0
    
    t = np.linspace(tstart, tend)
    results = odeint(f, y0, (a, b))

    plt.plot(t, results, label="Ergebnisse der gelösen ODE")
    plt.legend()
    plt.show()
    \end{minted}
\end{frame}


\subsection{Zugehöriges Reaktionsnetzwerk}
\begin{frame}[fragile]
    \frametitle{\insertsubsection}
    Welche Reaktionen laufen in der eben gelösten ODE ab?
    \[\dot{x} = f(x, t) = \alpha - \beta x\]
    \pause Erstellung von neuem Protein
    \[\xrightarrow{\alpha} Y\]
    \pause Degradation von Protein
    \[Y \xrightarrow{\beta}\emptyset\]
    \pause Insgesamt:
    \[\xrightarrow{\alpha} Y \xrightarrow{\beta}\emptyset\]
\end{frame}


\begin{frame}[fragile]
    Wie lautet eine ODE zu einem gegebenenen Reaktionsnetzwerk?
    \begin{align}
        A+A&\xrightarrow{\psi}B\\
        B+C&\xrightarrow{\phi}\emptyset
    \end{align}
    \begin{itemize}[<+->]
        \item Wie viele Komponenten haben wir?
        \item[ ] $\Rightarrow$ 3 Stück: $A, B, C$
        \item Betrachte zunächst Gleichung 2. Was passiert hier?
        \item[ ] $\Rightarrow$ $B$ und $C$ reagieren und werden vernichtet. 
        Die konzentration von $B$ und $C$ muss also kleiner werden.
        \item Mit welcher Rate passiert das?
        \pause \begin{align}
            \dot{B} &= -\phi BC\\
            \dot{C} &= -\phi BC
        \end{align}
    \end{itemize}
\end{frame}


\begin{frame}[fragile]
    Wie lautet eine ODE zu einem gegebenenen Reaktionsnetzwerk?
    \begin{align}
        A+A&\xrightarrow{\psi}B\\
        B+C&\xrightarrow{\phi}\emptyset
    \end{align}
    \begin{itemize}[<+->]
        \item Bisher haben wir:
        \pause \begin{align}
            \dot{B} &= -\phi BC\\
            \dot{C} &= -\phi BC
        \end{align}
        \item Betrachte jetzt noch Gleichung 1
        \item[ ] 2x Stoff $A$ wird umgewandelt zu $B$
        \item Wie lautet die Reaktions gleichung?
        \pause
        \begin{align}
            \dot{A} &= -2\psi A^2\\
            \dot{B} &= +2\psi A^2
        \end{align}
        \item Wie kombiniere ich nun diese Gleichungen?
    \end{itemize}
\end{frame}


\begin{frame}[fragile]
    \begin{itemize}
        \item[ ] Einzelne Komponenten aufaddieren und dann alles zusammenschreiben
        \pause
        \begin{align}
            \dot{A} &= -2\psi A^2\\
            \dot{B} &= +2\psi A^2 -\phi BC\\
            \dot{C} &= -\phi BC
        \end{align}
    \end{itemize}
\end{frame}


\begin{frame}[fragile]
    \begin{minted}[linenos, fontsize=\scriptsize, escapeinside=||]{python}
|\pause|from scipy.integrate import odeint
import matplotlib.pyplot as plt
import numpy as np

def f(y, t, k1, km1, k2):
    return (-2*k1*y[0]**2, 2*k1*y[0]**2 - k2*y[1]*y[2], - k2*y[1]*y[2])

|\pause|if __name__ == "__main__":
    tstart = 0.0
    tend = 10.0
    y0 = (1.0, 0.0, 0.5)
    k1 = 0.3
    k2 = 0.5
    t = np.linspace(tstart, tend)
    
|\pause|    results = odeint(f, y0, t, (k1, k2))

|\pause|    for i in range(results.shape[1]):
        plt.plot(t, results[:,i], label="Komponente " + str(i))
    plt.legend()
    plt.show()
    \end{minted}
\end{frame}


\subsection{Michaelis–Menten kinetics}
\begin{frame}
    \frametitle{\insertsubsection}
    Model findet weitläufig Verwendung:
    \begin{itemize}[<+->]
        \item Antigen–Antibody Binding
        \item DNA–DNA Hybridization
        \item Protein–Protein Interaction
    \end{itemize}
    \begin{align}
        A+B&\xrightleftharpoons[k_{-1}]{k_1}C\\
        C&\xrightarrow{k_2}A+D
    \end{align}
    Zugehörige ODEs:
    \begin{align}
        \dot{A} &= -k_1AB + k_{-1}C + k_2C\\
        \dot{B} &= -k_1AB + k_{-1}C\\
        \dot{C} &= +2k_1AB - 2k_{-1}C - 2k_2C\\
        \dot{D} &= +k_2C
    \end{align}
\end{frame}