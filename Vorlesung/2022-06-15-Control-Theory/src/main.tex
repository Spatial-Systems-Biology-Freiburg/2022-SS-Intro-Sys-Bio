\documentclass[10pt, compress]{beamer}

\usetheme{utopia}

\usepackage{booktabs}
\usepackage[scale=2]{ccicons}
\usepackage[outputdir=out]{minted}
\usepackage{natbib}

\usepgfplotslibrary{dateplot}

\usetikzlibrary{shapes,arrows,positioning,intersections}
\tikzstyle{block} = [rectangle, draw, fill=blue!20,
    text width=5em, text centered, rounded corners, minimum height=4em]
\tikzstyle{line} = [draw, -latex']


\usemintedstyle{trac}

\hypersetup{colorlinks=false}

\title{Control Theory in Biology}
\subtitle{}
\date{15.06.2022}
\author{Jonas Pleyer}
\institute{Freiburg Center for Data Analysis and Modeling (FDM)}

\begin{document}

\maketitle

\section{Introduction}
\label{sec:introduction}
\subsection{Historic Overview}
\label{subsec:introduction-history}
\begin{frame}{\insertsubsection}
	
\end{frame}
%
%
\begin{frame}{\insertsubsection}
	"We argue that the key properties of biochemical networks should be robust in order to ensure their proper functioning."~\cite{Barkai1997}
\end{frame}
%
%
\section{Examples from Biology}
\label{sec:examples}
\subsection{Child Birth}
\label{subsec:examples-pregnancy}
\begin{frame}{\insertsubsection}
    \begin{figure}
        \includegraphics[width=\textwidth]{media/Childbirth.jpg}
        \caption{Childbirth~\cite{albert2022}}
    \end{figure}
    \begin{itemize}[<+->]
        \item More pressure 
        \item[ ] $\rightarrow$ More contractions
        \item This is a \emph{positive feedback loop}.
    \end{itemize}
\end{frame}
%
%
\subsection{Temperature Regulation}
\label{subsec:examples-temperature}
\begin{frame}{\insertsubsection}
    \begin{minipage}[t]{0.349\textwidth}
        \begin{itemize}[<+->]
            \item More Sweat
            \item[ ] $\rightarrow$ Less Temperature
            \item More ...
            \item[ ] $\rightarrow$ Less ...
            \item This is a \emph{negative feedback loop}.
        \end{itemize}
    \end{minipage}%
    \begin{minipage}[t]{0.649\textwidth}
        \begin{figure}
            \centering
            \includegraphics[width=\textwidth]{media/Temperature-Regulation.jpg}
            \caption{Temperature regulation~\cite{albert2022}}
        \end{figure}
    \end{minipage}%
\end{frame}
%
%
\subsection{More Examples}
\label{subsec:more-examples}
\begin{frame}{\insertsubsection}

    \begin{itemize}[<+->]
        \item Tight calcium regulation in humans
        \item Receptor Networks
        \item Synthetic Biology
        \item Optogenetics (What we also do)
        \item Financial Markets
        \item Social Relationships
        \item ...
    \end{itemize}
\end{frame}
%
%
\section{Concepts}
\label{sec:concepts}
\subsection{Closed and Open Loop}
\label{subsec:concepts-closed-open-loop}
\begin{frame}{\insertsubsection}
	
\end{frame}
\section{Biology again}
\label{sec:biology-again}
\subsection{Recap}
\label{subsec:}
\begin{frame}{\insertsubsection}
    "We argue that the key properties of biochemical networks should be robust in order to ensure their proper functioning." - N. Barkai~\cite{Barkai1997}
    \begin{itemize}[<+->]
        \item These findings were simple integral control feedback loop.
        \item Robustness results from systems controlling themselves
        \item Feedback loops and control mechanisms are unavoidable in modern biology
    \end{itemize}
\end{frame}
%
%
% TODO mention non-linear systems are a challenge
\subsection{Further Information}
\begin{frame}{\insertsubsection}
    \begin{itemize}[<+->]
        \item Noise can play important role in stabilizing systems
        \item Almost all biological systems are non-linear (eg. Toggle-Switch) $\rightarrow$ some control-schemes do not work
        \item 
    \end{itemize}
\end{frame}

% \begin{frame}[fragile]
%   \frametitle{Sections}
%   Sections group slides of the same topic
% 
%   \begin{minted}[fontsize=\small]{latex}
%     \section{Elements}
%   \end{minted}
% 
%   for which the \emph{mtheme} provides a nice progress indicator \ldots
% \end{frame}
% 
% \section{Elements}
% 
% \begin{frame}[fragile]
%   \frametitle{Typography}
%       \begin{minted}[fontsize=\small]{latex}
% The theme provides sensible defaults to \emph{emphasize}
% text, \alert{accent} parts or show \textbf{bold} results.
%       \end{minted}
% 
%   \begin{center}becomes\end{center}
% 
%   The theme provides sensible defaults to \emph{emphasize} text,
%   \alert{accent} parts or show \textbf{bold} results.
% \end{frame}
% \begin{frame}{Lists}
%   \begin{columns}[onlytextwidth]
%     \column{0.5\textwidth}
%       Items
%       \begin{itemize}
%         \item Milk \item Eggs \item Potatos
%       \end{itemize}
% 
%     \column{0.5\textwidth}
%       Enumerations
%       \begin{enumerate}
%         \item First, \item Second and \item Last.
%       \end{enumerate}
%   \end{columns}
% \end{frame}
% \begin{frame}{Descriptions}
%   \begin{description}
%     \item[PowerPoint] Meeh.
%     \item[Beamer] Yeeeha.
%   \end{description}
% \end{frame}
% \begin{frame}{Animation}
%   \begin{itemize}[<+- | alert@+>]
%     \item \alert<4>{This is\only<4>{ really} important}
%     \item Now this
%     \item And now this
%   \end{itemize}
% \end{frame}
% \begin{frame}{Tables}
%   \begin{table}
%     \caption{Largest cities in the world (source: Wikipedia)}
%     \begin{tabular}{lr}
%       \toprule
%       City & Population\\
%       \midrule
%       Mexico City & 20,116,842\\
%       Shanghai & 19,210,000\\
%       Peking & 15,796,450\\
%       Istanbul & 14,160,467\\
%       \bottomrule
%     \end{tabular}
%   \end{table}
% \end{frame}
% \begin{frame}{Blocks}
% 
%   \begin{block}{This is a block title}
%     This is soothing.
%   \end{block}
% 
% \end{frame}
% \begin{frame}{Math}
%   \begin{equation*}
%     e = \lim_{n\to \infty} \left(1 + \frac{1}{n}\right)^n
%   \end{equation*}
% \end{frame}
% \begin{frame}{Line plots}
%   \begin{figure}
%     \begin{tikzpicture}
%       \begin{axis}[
%         mlineplot,
%         width=0.9\textwidth,
%         height=6cm,
%       ]
% 
%         \addplot {sin(deg(x))};
%         \addplot+[samples=100] {sin(deg(2*x))};
% 
%       \end{axis}
%     \end{tikzpicture}
%   \end{figure}
% \end{frame}
% \begin{frame}{Bar charts}
%   \begin{figure}
%     \begin{tikzpicture}
%       \begin{axis}[
%         mbarplot,
%         xlabel={Foo},
%         ylabel={Bar},
%         width=0.9\textwidth,
%         height=6cm,
%       ]
% 
%       \addplot plot coordinates {(1, 20) (2, 25) (3, 22.4) (4, 12.4)};
%       \addplot plot coordinates {(1, 18) (2, 24) (3, 23.5) (4, 13.2)};
%       \addplot plot coordinates {(1, 10) (2, 19) (3, 25) (4, 15.2)};
% 
%       \legend{lorem, ipsum, dolor}
% 
%       \end{axis}
%     \end{tikzpicture}
%   \end{figure}
% \end{frame}
% \begin{frame}{Quotes}
%   \begin{quote}
%     Veni, Vidi, Vici
%   \end{quote}
% \end{frame}
% 
% 
% \section{Conclusion}
% 
% \begin{frame}{Summary}
% 
%   Get the source of this theme and the demo presentation from
% 
%   \begin{center}\url{github.com/matze/mtheme}\end{center}
% 
%   The theme \emph{itself} is licensed under a
%   \href{http://creativecommons.org/licenses/by-sa/4.0/}{Creative Commons
%   Attribution-ShareAlike 4.0 International License}.
% 
%   \begin{center}\ccbysa\end{center}
% 
% \end{frame}
% 
\plain{Questions?}

\bibliographystyle{alpha}
\bibliography{ControlTheory}
% 
\end{document}
