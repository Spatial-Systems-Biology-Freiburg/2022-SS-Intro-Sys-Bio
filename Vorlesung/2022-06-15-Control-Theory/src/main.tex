\documentclass[10pt, compress]{beamer}

\usetheme{utopia}

\usepackage{booktabs}
\usepackage[scale=2]{ccicons}
\usepackage[outputdir=out]{minted}
\usepackage{natbib}

\usepgfplotslibrary{dateplot}

\usetikzlibrary{shapes,arrows,positioning,intersections}
\tikzstyle{block} = [rectangle, draw, fill=blue!20,
    text width=5em, text centered, rounded corners, minimum height=4em]
\tikzstyle{line} = [draw, -latex']


\usemintedstyle{trac}

\hypersetup{colorlinks=false}

\title{Control Theory in Biology}
\subtitle{}
\date{15.06.2022}
\author{Jonas Pleyer}
\institute{Freiburg Center for Data Analysis and Modeling (FDM)}

\begin{document}

\maketitle

\section{Introduction}
\label{sec:introduction}
\subsection{Historic Overview}
\label{subsec:introduction-history}
\begin{frame}{\insertsubsection}
	
\end{frame}
%
%
\begin{frame}{\insertsubsection}
	"We argue that the key properties of biochemical networks should be robust in order to ensure their proper functioning."~\cite{Barkai1997}
\end{frame}
%
%
\section{Examples from Biology}
\label{sec:examples}
\subsection{Calcium Regulation in Mammals}
\begin{frame}{\insertsubsection}

\end{frame}
%
%
\section{Concepts}
\label{sec:concepts}
\subsection{Controller Types}
\label{subsec:concepts-pid-controllers}
\begin{frame}{\insertsubsection}
	
\end{frame}
%
%
\begin{frame}{\insertsubsection}
\begin{figure}
	\begin{tikzpicture}
		% Draw nodes
		\node [block] (controller) {Controller};
		\coordinate [left of=controller, node distance=3cm] (init);
		\node [block, right of=controller, node distance=5cm] (system) {System};
		\coordinate [right of=system, node distance=3cm] (end);
		% Draw lines connecting (arrows)
		\path [line] (init) -- node [midway, above] {Input} (controller);
		\path [line] (controller) -- node [midway, above] {Control Signal} (system);
		\path [line] (system) -- node [midway, above] {Output} (end);
	\end{tikzpicture}
	\caption{Open Loop Control System}
\end{figure}
\end{frame}
%
%
\begin{frame}{\insertsubsection}
\begin{figure}
	\begin{tikzpicture}
		% Draw nodes
		\node [block] (controller) {Controller};
		\node [block, right=3cm of controller] (system) {System};
		\node [block, below right=0.5cm and 0.5cm of controller] (measurement) {Measurement};
		% Helper coordinates
		\coordinate [left=2cm of controller] (init);
		\coordinate [right=2cm of system] (end);
		\coordinate [right=1cm of system] (measurepoint);
		\coordinate [below right=0.5cm and 5cm of controller] (rightcorner);
		\coordinate [below of=controller] (leftcorner);
		% Draw lines connecting (arrows)
		\path [line] (init) -- node [midway, above] {Input} (controller);
		\path [line] (controller) -- node [midway, above] {Control Signal} (system);
		\path [line] (system) -- node [midway, above] {Output} (end);
		% \path [line] (measurepoint) -- (rightcorner) -- (measurement);
		%\draw (measurepoint) to (rightcorner) to (measurement) to (leftcorner) to (controller);
		\path [line] (measurepoint) |- (measurement);
		\path [line] (measurement) -| (controller);
	\end{tikzpicture}
	\caption{Closed Loop Control System}
\end{figure}
\end{frame}
%
%
%
%
\subsection{Combinations of Controller}
\label{subsec:concepts-combinations}
\begin{frame}{\insertsubsection}
	
\end{frame}
% \begin{frame}[fragile]
%   \frametitle{Sections}
%   Sections group slides of the same topic
% 
%   \begin{minted}[fontsize=\small]{latex}
%     \section{Elements}
%   \end{minted}
% 
%   for which the \emph{mtheme} provides a nice progress indicator \ldots
% \end{frame}
% 
% \section{Elements}
% 
% \begin{frame}[fragile]
%   \frametitle{Typography}
%       \begin{minted}[fontsize=\small]{latex}
% The theme provides sensible defaults to \emph{emphasize}
% text, \alert{accent} parts or show \textbf{bold} results.
%       \end{minted}
% 
%   \begin{center}becomes\end{center}
% 
%   The theme provides sensible defaults to \emph{emphasize} text,
%   \alert{accent} parts or show \textbf{bold} results.
% \end{frame}
% \begin{frame}{Lists}
%   \begin{columns}[onlytextwidth]
%     \column{0.5\textwidth}
%       Items
%       \begin{itemize}
%         \item Milk \item Eggs \item Potatos
%       \end{itemize}
% 
%     \column{0.5\textwidth}
%       Enumerations
%       \begin{enumerate}
%         \item First, \item Second and \item Last.
%       \end{enumerate}
%   \end{columns}
% \end{frame}
% \begin{frame}{Descriptions}
%   \begin{description}
%     \item[PowerPoint] Meeh.
%     \item[Beamer] Yeeeha.
%   \end{description}
% \end{frame}
% \begin{frame}{Animation}
%   \begin{itemize}[<+- | alert@+>]
%     \item \alert<4>{This is\only<4>{ really} important}
%     \item Now this
%     \item And now this
%   \end{itemize}
% \end{frame}
% \begin{frame}{Tables}
%   \begin{table}
%     \caption{Largest cities in the world (source: Wikipedia)}
%     \begin{tabular}{lr}
%       \toprule
%       City & Population\\
%       \midrule
%       Mexico City & 20,116,842\\
%       Shanghai & 19,210,000\\
%       Peking & 15,796,450\\
%       Istanbul & 14,160,467\\
%       \bottomrule
%     \end{tabular}
%   \end{table}
% \end{frame}
% \begin{frame}{Blocks}
% 
%   \begin{block}{This is a block title}
%     This is soothing.
%   \end{block}
% 
% \end{frame}
% \begin{frame}{Math}
%   \begin{equation*}
%     e = \lim_{n\to \infty} \left(1 + \frac{1}{n}\right)^n
%   \end{equation*}
% \end{frame}
% \begin{frame}{Line plots}
%   \begin{figure}
%     \begin{tikzpicture}
%       \begin{axis}[
%         mlineplot,
%         width=0.9\textwidth,
%         height=6cm,
%       ]
% 
%         \addplot {sin(deg(x))};
%         \addplot+[samples=100] {sin(deg(2*x))};
% 
%       \end{axis}
%     \end{tikzpicture}
%   \end{figure}
% \end{frame}
% \begin{frame}{Bar charts}
%   \begin{figure}
%     \begin{tikzpicture}
%       \begin{axis}[
%         mbarplot,
%         xlabel={Foo},
%         ylabel={Bar},
%         width=0.9\textwidth,
%         height=6cm,
%       ]
% 
%       \addplot plot coordinates {(1, 20) (2, 25) (3, 22.4) (4, 12.4)};
%       \addplot plot coordinates {(1, 18) (2, 24) (3, 23.5) (4, 13.2)};
%       \addplot plot coordinates {(1, 10) (2, 19) (3, 25) (4, 15.2)};
% 
%       \legend{lorem, ipsum, dolor}
% 
%       \end{axis}
%     \end{tikzpicture}
%   \end{figure}
% \end{frame}
% \begin{frame}{Quotes}
%   \begin{quote}
%     Veni, Vidi, Vici
%   \end{quote}
% \end{frame}
% 
% 
% \section{Conclusion}
% 
% \begin{frame}{Summary}
% 
%   Get the source of this theme and the demo presentation from
% 
%   \begin{center}\url{github.com/matze/mtheme}\end{center}
% 
%   The theme \emph{itself} is licensed under a
%   \href{http://creativecommons.org/licenses/by-sa/4.0/}{Creative Commons
%   Attribution-ShareAlike 4.0 International License}.
% 
%   \begin{center}\ccbysa\end{center}
% 
% \end{frame}
% 
\plain{Questions?}

\bibliographystyle{alpha}
\bibliography{ControlTheory}
% 
\end{document}
